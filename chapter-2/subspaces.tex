\section{Subspaces}

The discussion of Problem 21 established that the four axioms that define vector spaces (the vector and scalar distributive laws, the associative laws, and the scalar identity) are independent. According to the official definition, therefore, a vector space is an abelian group $\V$ on which a field $\F$ "acts" so that the four independent axioms are satisfied. The set $\P$ of all polynomials with, say, real coefficients, is an example of a real vector space, and so is the subset $\P_3$ of all polynomials of degree less than or equal to 3. The set $\R^3$ of all ordered tuples of real number is a real vector space, and so is the set $\V$ consisting of all ordered triples with sum zero. (See Problem 20.) The subset $\Q^3$ of $\R^3$ consisting of all triples with rational coordinates is not a vector space over $\R$ (an irrational number times an element of $\Q^3$ might not belong to $\Q^3$), and the subset $\mathbf{X}$ of $\R^3$ consisting of all triples with at least one coordinate equal to $0$ is not a vector space (the sum of two elements of $\mathbf{X}$ does not necessarily belong to $\mathbf{X}$).

Those examples illustrate and motivate an important definition: a non-empty subset $\mathbf{M}$ of a vector space $\V$ is a \textbf{subspace} of $\V$ if the sum of two vectors in $\mathbb{M}$ is always in $\mathbf{M}$ and if the product of a vector in $\mathbf{M}$ with every scalar is always in $\mathbf{M}$. An equivalent way of phrasing the definition is this: a non-empty set $\mathbf{M}$ is a subspace if and only if $\alpha x + \beta y$ belongs to $\mathbf{M}$ whenever $x$ and $y$ are vectors in $\mathbf{M}$ and $\alpha$ and $\beta$ are arbitrary scalars, or, in other owrds, subspaces are just the non-empty subsets closed under the formation of linear combinations.

The set $\mathbf(O)$ consisting of the vector 0 alone is a subspace of every vector space $\V$ (it is usually referred to as the \textbf{trivial} subspace - the others are called non-trivial), and so is the entire space $\V$. The way the word "subset" and "subspace" are used is intended to allow these extremes. (A subspace of $\V$ different from $\V$ is called a \textbf{proper} subspace - in this language $\V$ itself is called the improper subspace.) To get more interesting examples of subspaces, it's a good idea to enlarge the stock of examples of vector spaces.

It has already been noted (see Solution 20) that every field is a vector space over itself. In particular, $\R$ is a vector space over $\R$, but, this is more interesting, $\R$ is a vector space over $\Q$ also - just forget how to multiply real numbers by anything except rational numbers. In this situation, where $\R$ is regarded as a rational vector space, the subset $\Q$ of $\R$ is a new example of a subspace, and so is the larger subset $\Q(\sqrt{2})$ (see Problem 16). In the same spirit, $\C$ (with the operation of addition) is a vector space over $\C$, and it is also a vector space over $\R$; from the latter point of view, the set $\R$ is a subspace (a real subspace of $\C$).

Usually when vector spaces are discussed a field $\F$ has been fixed once and for all, and it is clear that all vector spaces under consideration are over $\F$. If, however, there is some chance that the underlying field may have to be changed during the discussion, then it is necessary to specify the field each time. One way to do that is to speak of an $\F$ vector space; this is the general form of speaking of rational, real, and complex vector spaces.

If $\F$ is a field and $n$ is a positive integer, then the set of all $n$-tuples $(\xi_1, \xi_2, ..., \xi_n)$ of elements of $\F$, is an $\F$ vector space (the addition of the $n$-tuples that play the role of evectors is coordinate by coordiante, and so is multiplication by an element of $\F$); this space is denoted by $\F^n$. The set of those $n$-tuples whose first coordinate is equal to $0$ is a subspace.

Here is an important non-trivial example: the set of all real-valued functions on, say, a closed interval is a vector space over $\R$ if vector addition and scalar multiplication are defined in the obvious pointwise fashion. The set of all continuous functions is an example of a subspace of that space. A different generalization of $\R^n$ is the set of all infinite sequences $\left\{ \xi_1, \xi_2, \xi_3,... \right\}$, of real numbers; an example of a subspace is the subset consisting of all those sequences for which series $\sum_{n=1}^{\infty} \xi_n$ is convergent, and a subspace of that subspace is the subset of all those sequences for which the series is absolutely convergent.

For examples with a more geometric flavor, consider the real vector space $\R^2$ and in it the subset $\mathbf{M}$ of all vectors of the form $(\alpha, 2\alpha)$, where $\alpha$ is an arbitrary real number. Equivalently, $\mathbf{M}$ consists of the vectors whose second coordinate is equal to twice the first; in the usual language of analytic geometry the elements of $\mathbf{M}$ are the points on the line through the origin with slope $2$; the line described by the equation $y=2x$. (Examples like this are the reason why linear algebra is called linear: the expression refers to the algebra of lines and their natural higher-dimensional generalizations.) The example is typical: eveyr straight line through the origin is a subspace of $\R^2$, and every non-trivial proper subspace of $\R^2$ is like that. The generalization of these examples to $\R^3$ is straightforward: the non-trivial proper subspaces of $\R^3$ are the liens and planes through the origin.

What's special about the origin? Answer: it necessarily belongs to every subspace. Proof: if $\mathbf{M}$ is a subspace, then if $x$ is an arbitrary element of $\mathbf{M}$, then $0 \cdot x$ belongs to $\mathbf{M}$ (scalar multiples), and since it is already known that $0 \cdot x = 0$, it follows that $0 \in \mathbf{M}$ for all $\mathbf{M}$. The definition of subspaces could have been formulated this way: a subset $\mathbf{M}$ of $\V$ is a subspace if $\mathbf{M}$ itself is a vector space with respect to the same the linear operations (vector addition and scalar multiplication, or, in on phrase, linear combination) as are given in $\V$. Since every vector space contains its zero vector, the presence of $0$ in $\mathbf{M}$ should not come as a surprise.

\begin{problem}
\begin{enumerate}[(a)]
    \item Consider the complex vector space $\C^3$ and the subsets $\mathbf{M}$ of $\C^3$ consisting of those vectors $(\alpha,\beta,\gamma)$, for which

          \begin{enumerate}[(1)]
              \item $\alpha = 0$,
              \item $\beta = 0$,
              \item $\alpha + \beta = 1$,
              \item $\alpha + \beta = 0$,
              \item $\alpha + \beta \geq 0$,
              \item $\alpha$ is real.
          \end{enumerate}

          In which of these cases is $\mathbf{M}$ a subspace of $\C^3$?

    \item Consider the complex vector space $\P$ and the subsets $\mathbf{M}$ of all those vectors (polynomials) $p$ for which

          \begin{enumerate}[(1)]
              \item $p$ has degree $3$,
              \item $2p(0) = p(1)$,
              \item $p(t) \geq 0$ whenever $0 \leq t \leq 1$,
              \item $p(t) = p(t-1)$ for all $t$.
          \end{enumerate}

          In which of these cases is $\mathbf{M}$ a subspace of $\P$?
\end{enumerate}
\end{problem}

